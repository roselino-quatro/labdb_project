\section{Arquitetura Desenvolvida}
\subsection{Tecnologias Utilizadas}
\subsubsection{Docker e Docker Compose}
Para desenvolvimento e transportabilidade da plataforma utilizamos containerização e orquestração utilizando Docker e Docker Compose,
essas ferramentas nos permitem manter um sistema uniformizado e realizar pipelines de deploy e teste de forma eficiente.

Os detalhes dos serviços criados e configurações escolhidas podem ser encontrados no arquivo "docker-compose.yml"

\subsubsection{PostgreSQL}
O Sistema Gerenciador de Banco de Dados escolhido foi o PostgreSQL, tanto pela sua simplicidade de uso, custo de processamento leve e conhecimento prévio do grupo. 

\subsubsection{Flask}
Para o desenvolvimento do backend utilizamos a biblioteca Flask do Python, apesar de possuir limitações comparadas a outras opções ela é leve
e permitiu um desenvolvimento rápido e efetivo até entre os membros do gruopo sem experiência nela.

\subsubsection{NextJs}
O Frontend da plataforma foi desenvolvido utilizando o framework NextJs, mesmo o foco do projeto não sendo a interface de usuário temos interesse em
manter um frontend legivel e bonito mantendo um desenvolvimento rápido e integração com o backend, facilmente provida pelo uso de componentes React

\subsection{Lógica Interna}
Realizamos a escolha de, sempre que possível, separar o código SQL do código de outras linguagens, tanto por uma questão de modularização
e organização quanto para evidenciar como os conceitos da disciplina foram aplicados no desenvolvimento.

Dentro do diretório "sql" criamos um script para gerar o schema completo da base, o código de inicialização em python verifica a existência da
base de dados e automaticamente executa esse script, depois ele também verifica se a base já está populada, se não ele executa ordenamente
a geração de dados sintéticos para o uso da base.

Dentro deste também temos o subdiretório "funcionalidades", com scripts separados por entidade relacionada que criam as funções, procedures
e triggers que vão ser utilizados no código. O subdiretório "queries" contem as chamadas de queries que são puxadas pelo backend e acionadas 
utilizando a passagem de paramêtros, dessa forma conseguimos isolar as diferentes lógicas do código entre backend e banco.

